

 layout\+: post title\+: \char`\"{}\+Welcome to Jekyll!\char`\"{} date\+: 2020-\/10-\/07 10\+:20\+:10 +0200 \subsection*{categories\+: jekyll update }

You’ll find this post in your {\ttfamily \+\_\+posts} directory. Go ahead and edit it and re-\/build the site to see your changes. You can rebuild the site in many different ways, but the most common way is to run {\ttfamily jekyll serve}, which launches a web server and auto-\/regenerates your site when a file is updated.

To add new posts, simply add a file in the {\ttfamily \+\_\+posts} directory that follows the convention {\ttfamily Y\+Y\+Y\+Y-\/\+M\+M-\/\+D\+D-\/name-\/of-\/post.\+ext} and includes the necessary front matter. Take a look at the source for this post to get an idea about how it works.

Jekyll also offers powerful support for code snippets\+:

\{\% highlight ruby \%\} def print\+\_\+hi(name) puts \char`\"{}\+Hi, \#\{name\}\char`\"{} end print\+\_\+hi(\textquotesingle{}Tom\textquotesingle{}) \#=$>$ prints \textquotesingle{}Hi, Tom\textquotesingle{} to S\+T\+D\+O\+UT. \{\% endhighlight \%\}

Check out the \href{http://jekyllrb.com/docs/home}{\tt Jekyll docs} for more info on how to get the most out of Jekyll. File all bugs/feature requests at \href{https://github.com/jekyll/jekyll}{\tt Jekyll’s Git\+Hub repo}. If you have questions, you can ask them on \href{https://talk.jekyllrb.com/}{\tt Jekyll Talk}. 